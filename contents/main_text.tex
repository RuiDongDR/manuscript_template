\section{Introduction}
The introduction provides context for the study. You can cite papers using the \texttt{biblatex} system like this: \cite{wang2020simple}. Use the introduction to define the problem and state your primary contributions.

\section{Results}
This section describes the findings of the study. 

\subsection{Mathematical Notation and Bold Vectors}
We can use the custom macros for bold vectors and matrices. For example, let $\bx$ be our input vector and $\bbeta$ be the vector of coefficients. The linear model is often written as:
\begin{equation}
    y = X \beta + \epsilon
\end{equation}
Note how we use \hf{highlighter colors} to draw attention to specific terms or critical results during the drafting phase.

\subsection{Data Analysis and Observations}
\label{sec:obs}
This is a subsection for detailed analysis. 

\rmk{Remark: You can use the remark command to leave meta-comments about the narrative flow, such as reminding yourself that this section needs to link back to the methodology.}

We can refer back to specific equations or sections easily. For instance, \cref{sec:obs} contains our primary observations.

\subsection{Collaborative Notes}
During the writing process, different authors can leave feedback directly in the text:
\begin{itemize}
    \item \inda{Author A: We should double-check the p-values in this paragraph.}
    \item \indb{Author B: I have updated the figure to reflect the new dataset.}
    \item \done{This paragraph has been proofread and is ready for submission.}
\end{itemize}

\section{Discussion}
The discussion interprets the results in the context of the broader field. 

You may want to highlight the limitations of the current study or suggest directions for future work. If you have a complex thought that requires a specific call-out, the remark command remains useful:
\rmk{Final Check: Ensure all citations in this section are consistent with the main bibliography.}